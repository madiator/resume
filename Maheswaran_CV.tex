%______________________________________________________________________________________________________________________
% @brief    LaTeX2e Resume for Maheswaran Sathiamoorthy
\documentclass[margin,line]{resume}
\usepackage{hyperref}
%\usepackage[left=1cm,top=1cm,right=1cm,nohead,nofoot]{geometry}

%______________________________________________________________________________________________________________________
\begin{document}
\name{\Large Maheswaran Sathiamoorthy}
\begin{resume}

%__________________________________________________________________________________________________________________
% Contact Information

\section{\mysidestyle Contact\\Information}


\begin{tabular}{@{}p{3in}p{6in}}
RTH 419, 3710 McClintock Ave & {\it Voice:}  +1 323-610-5440 \\            
University of Southern California & {\it E-mail:}  msathiam at usc.edu\\
Los Angeles, CA-90089 & {\it WWW:} \href{http://anrg.usc.edu/~maheswaran}{http://anrg.usc.edu/$\sim$maheswaran} \\     
\end{tabular}

    %__________________________________________________________________________________________________________________
    % Research Interests
    \section{\mysidestyle Research\\Interests}
	Cloud Computing: Coding techniques for distributed storage. \\
	Vehicular Networks: Analysis and design of content distribution strategies, primarily relying on coded storage; design of protocols for inter-vehicular file transfers.


    %__________________________________________________________________________________________________________________
    % Education
    \section{\mysidestyle Education}

    \textbf{University of Southern California}, Los Angeles, California, USA \\%\vspace{2mm}\\\vspace{1mm}%
    \textsl{Doctor of Philosophy, Electrical Engineering} \hfill \textbf{ Aug 2008 -- Present}\vspace{-3mm}\\\vspace{-1mm}
    \begin{list2}
        \item GPA: 3.93/4
        \item Advisors:  Prof. Bhaskar Krishnamachari \& Prof. Alexandros G. Dimakis
    \end{list2}\vspace{-1.5mm}
    
    
    \textbf{Indian Institute of Technology (IIT)}, Kharagpur, West Bengal, India \\%\vspace{2mm}\\\vspace{1mm}%
    \textsl{B.Tech(H), Electronics and Electrical Communication Engineering} \hfill \textbf{ July 2004 -- May 2008}\vspace{-3mm}\\\vspace{-1mm}%
    \begin{list2}
    		\item GPA:	9.27 out of 10.00
    		\item Ranked 3/50 in the department and 7/650 in the Institute        
    \end{list2}\vspace{-1.5mm}%


\section{\mysidestyle Publications}
\begin{list2}
    \item Maheswaran Sathiamoorthy, Wei Gao, Bhaskar Krishnamachari, Guohong Cao
    ``Minimum Latency Data Diffusion in Intermittently Connected Mobile Networks'', 
    to appear in \textsl{2012 IEEE 75th Vehicular Technology Conference: VTC2012-Spring, 6-9 May 2012, Yokohama, Japan.}
    \item Maheswaran Sathiamoorthy, Alexandros G. Dimakis, Bhaskar Krishnamachari, Fan Bai, 
    ``Distributed Storage Codes Reduce Latency in Vehicular Networks'', 
    to appear in \textsl{Proceedings of the IEEE INFOCOM Mini-conference, 2012.}
    \item Majed Alresaini, Maheswaran Sathiamoorthy, Bhaskar Krishnamachari, Michael J. Neely, ``Backpressure with Adaptive Redundancy (BWAR)'', 
    to appear in \textsl{Proceedings of the IEEE INFOCOM, 2012.}
    \item Sangwon Lee, Sundeep Pattem, Maheswaran Sathiamoorthy, Bhaskar Krishnamachari, Antonio Ortega, ``Spatially-Localized Compressed Sensing and Routing in Multi-Hop Sensor Networks'', \textsl{3rd International Conference on Geosensor Networks}, July 2009, Pages 11-20.  
    \end{list2}

\section{\mysidestyle Projects}

  \textbf{VANETSim: A Vehicular Network Simulator} \hfill \textbf{Jan 2012 -- Present}
  \begin{list2}
   \item A Java based simulator primarily designed to study coded and uncoded content distribution strategies in vehicular networks.
   \item Open sourced at \href{https://github.com/madiator/VANETSim}{https://github.com/madiator/VANETSim}
   \item Advised by Prof. Bhaskar Krishnamachari in collaboration with General Motors R\&D.
  \end{list2}

  \textbf{Implementation of Regenerating Codes for Hadoop} \hfill \textbf{Sept 2011 -- Present}
  \begin{list2}
   \item Regenerating codes specially designed for data centers are being implemented over Hadoop HDFS
   \item Based on Facebook's implementation of HDFS-RAID
   \item Open sourced at \href{https://github.com/madiator/hadoop-20}{https://github.com/madiator/hadoop-20}
   \item About 2x reduction in network utilization and disk I/O during file repairs.
   \item Advised by Prof. Alex Dimakis. 
  \end{list2}
  
  \textbf{Network Coding for Vehicular Networks} \hfill \textbf{Aug 2010 -- Present}
  \begin{list2}
   \item Network Coding applied to Vehicular Networks to minimize the delay in content retrieval. 
   \item Obtained theoretical upper bounds on delay and showed regions where Network Coding performs better than naive distribution strategy.
   \item Simulated on real taxi traces to show the improvement.
   \item Continued the work at General Motors to test on real vehicles. 
   \item Advised by Prof. Bhaskar Krishnamachari and Prof. Alex Dimakis. 
  \end{list2}
  
  \textbf{Twitter Retweet Dynamics} \hfill \textbf{Apr 2010 -- Present}
  \begin{list2}
   \item Using data collected from Twitter, we showed an interesting trend between the number of retweets received and the probability of retweeting by Twitter users. 
   \item Work was presented at the \textsl{2nd Annual Annenberg Symposium} and was invited to present again at the \textsl{3rd Annual Annenberg Reception}.
\newpage
   \item Future work involves characterization of Tweets based on their space-time properties, spam detection in Twitter etc. 
   \item Advised by Prof. Antonio Ortega.
  \end{list2}
  

  \textbf{Social-aware Data Diffusion} \hfill \textbf{Sep 2010 -- Oct 2011}
  \begin{list2}
   \item Some of the current data diffision strategies involve using mobility information to decide which data to diffuse. 
   \item Our work involves overlaying the social structure on top of the mobility information to perform social-aware data diffusion.
   \item Advised by Prof. Bhaskar Krishnamachari and Prof. Guohong Cao, PSU.
  \end{list2}

  \textbf{Implementation of MapReduce} \hfill \textbf{Feb 2010 -- Apr 2010}
  \begin{list2}
  \item Implemented MapReduce (a framework for distributed processing) on USC's High-Performance Computing and Communications (HPCC) cluster consisting of thousands of nodes, as part of a course. 
  \item Used it to study huge network datasets. 
  \item Course (Computer Communications, CS551) advised by Prof. John Heidemann. 
  \end{list2}
 
  \textbf{Other projects} 
  \begin{list2}   
   \item Implementation of Nachos, a software that simulates a small OS, involving process management, memory management, interprocess communication, fault tolerance etc. for a course on Operating Systems.
   \item Flash Scheduling - proposed and analyzed a new scheduling algorithm in a multi-user communication system with varying number of users (term project for Computer Communications course).
  \end{list2}


\section{\mysidestyle Internship\\Experience}

    %%\textbf{University of Southern California}, Los Angeles    				 \vspace{2mm}\\\vspace{1mm}%
    %%\textsl{Graduate Student} \hfill \textbf{August 2008 -- present} \vspace{-3mm}\\\vspace{-1mm} \\  
    %%Over the period of around three months, I collected graph structure of Twitter comprising about two million nodes. This will be used to study the network structure of Twitter. Another aspect I investigated was an edge-evolution model on a fixed number of nodes in an Erd�s�R�nyi random graph where edges are added and deleted based on the similarity between nodes. A problem I recently started to look into is Contact Processes, where I try to fit models (markov models to begin with) for state changes of humans so as to match the observed contact processes between individuals.


    \textbf{General Motors R\&D}, Warren, MI\\
    \textsl{Summer Internship} \hfill \textbf{May 2011 -- Aug 2011}
    \begin{list2}
    \item Developed a new inter-vehicular video sharing application based on GM's existing Wavecast system for vehicular communication.
    \item Integrated erasure coding into the application for faster distributed file downloads. 
    \item Additionally, developed an Android application to act as the front end (which connects to and controls the Linux based video sharing application wirelessly).
    \end{list2}
    

    \textbf{University of Southern California}, Los Angeles\\%\vspace{1mm}\\\vspace{1mm}%
    \textsl{Summer Internship} \hfill \textbf{May 2007 -- July 2007}
    \begin{list2}
    \item Worked on the energy reduction of Wireless Sensor Networks using Compressed Sensing
    \item Compressed Sensing is used to integrate compression and sensing to achieve energy gains as high as 90\% in ideal conditions.
    \item The work involved using wavelet and DCT domains to perform compressed sensing to sense temperature. Sparse random projections were used to effectively reduce the number of communications to the sink. Methods for optimizing the projections were also investigated.     
    \end{list2}
    
    \textbf{Nanyang Technological University}, Singapore \\%\vspace{1mm}\\\vspace{1mm}%
    \textsl{Summer Internship} \hfill \textbf{May 2006 -- July 2006}
    \begin{list2}
    \item Developed an English Continuous Speech Recognizer based on TIMIT Database using HTK software.
    \item Automated (using Perl scripts) the process of training and testing the Speech Recognizer.
    \item Worked on Variable Frame Rate Algorithms and tested them on the CENSREC-3 database.
    \item Researched on Spectral Entropy based Speech Features and came up with modifications along with testing it on the CENSREC-3 database.
    \end{list2}
    

\section{\mysidestyle Teaching}
TA for \textbf{Wireless and Mobile Device Networks Design and Laboratory} (Spring 2012).

 \section{\mysidestyle Skills} 

    \textsl{Programming Languages}: C, C++, Java, Python\\
    \textsl{Software Experiences}: Hadoop, Android, TCP/IP, Matlab, \LaTeX, Processing\\
    \textsl{Databases}: MySQL, MongoDB\\
    \textsl{Scripting Languages}: Perl, Linux shell scripting \\
    \textsl{Hardware Description}: Verilog HDL \\
    \textsl{Web Technologies}: HTML, CSS, JavaScript, PHP \\
    \textsl{Foreign Languages}: German [beginner level]

%__________________________________________________________________________________________________________________
\newpage
 % Courses

\section{\mysidestyle Courses} 

%\begin{tabular}{@{}p{10cm}p{4cm}}
%Analysis of Algorithms       \\  Approximation Algorithms(ongoing)                   \\
%Design and Analysis of Computer Communication Networks                        \\
%Computer Communications (ongoing) \\ Queueing Theory                         \\
%Probabilistic Methods in Computer Systems Modeling    \\
%Operating Systems       \\  Wavelets        \\
%\end{tabular}

\begin{tabular}{@{}p{7cm}p{7cm}}
Analysis of Algorithms       &  Approximation Algorithms                   \\
Queueing Theory  & Operating Systems \\
Computer Communications & Wavelets \\
Random Processes in Engineering  & Distributed Storage Theory
\end{tabular} \\
Design and Analysis of Computer Communication Networks \\
Probabilistic Methods in Computer Systems Modeling    \\





%__________________________________________________________________________________________________________________
    % Honours and Awards
    
\section{\mysidestyle Honours and\\Awards} 
\begin{list2}
		\item USC Annenberg Fellowship 2008-2010	- University of Southern California
    \item InfoUSA Summer Fellowship 2007	- University of Southern California %\vspace{1mm}\\%   
	  \item One of the seven finalists of Trilogy's Pirates of the Corporate, a web 2.0 business plan contest held at Hong Kong. %\vspace{1mm}\\%       
	  \item Got Amateur Radio License (Grade II) from Ministry of Communications, Government of India in 2003.%\vspace{1mm}\\%
\item Awarded Rashtrapati Puraskar (President Award) as a Scout (Bharat Scouts and Guides) by the 10th President of India Shri K R Narayanan in 2002.%\vspace{1mm}\\%
\item Ranked among top 0.5\% of about 170,000 students appeared in JEE 2004. \\%\vspace{1mm}\\%
% \item Secured Rank 86 in TNPCEE (Tamilnadu Professional Courses Entrance Examination, the Common Entrance Test for Tamilnadu Engineering Colleges) out of more than 150,000 students.
\end{list2}
  



%______________________________________________________________________________________________________________________
\section{\mysidestyle References} 
Prof. Bhaskar Krishnamachari \\
Associate Professor \\
Ming Hsieh Department of Electrical Engineering \\
USC Viterbi School of Engineering, Los Angeles, CA \\
Email: bkrishna at usc.edu \\
\href{http://ceng.usc.edu/~bkrishna}{http://ceng.usc.edu/$\sim$bkrishna}

Prof. Alexandros G. Dimakis \\
Assistant Professor \\
Ming Hsieh Department of Electrical Engineering \\
USC Viterbi School of Engineering, Los Angeles, CA \\
Email: dimakis at usc.edu \\
\href{http://www-bcf.usc.edu/~dimakis}{http://www-bcf.usc.edu/$\sim$dimakis}

%______________________________________________________________________________________________________________________
\end{resume}
\end{document}



