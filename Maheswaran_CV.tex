%______________________________________________________________________________________________________________________
% @brief    LaTeX2e Resume for Maheswaran Sathiamoorthy
\documentclass[margin,line]{resume}
\usepackage{hyperref}
%\usepackage[left=1cm,top=1cm,right=1cm,bottom=1cm,nohead,nofoot]{geometry}

%______________________________________________________________________________________________________________________
\begin{document}
\name{\Large Maheswaran Sathiamoorthy}
\begin{resume}

%__________________________________________________________________________________________________________________
% Contact Information

\section{\mysidestyle Contact\\Information}


\begin{tabular}{@{}p{3in}p{6in}}
RTH 419, 3710 McClintock Ave &  {\it E-mail:}  msathiam at usc.edu\\
University of Southern California & {\it WWW:} \href{http://anrg.usc.edu/~maheswaran}{http://anrg.usc.edu/$\sim$maheswaran} \\
Los Angeles, CA-90089 &      \\
\end{tabular}

	%__________________________________________________________________________________________________________________
    % Research Interests
    \section{\mysidestyle Research\\Interests}
	Data Center Networks: Erasure coding techniques for distributed storage; reliable block storage. \\
	Traditional Networks: Analysis and design of content distribution strategies that primarily rely on coded storage, with a focus on Vehicular Networks.
	General: Distributed Storage, Distributed Systems, Big Data, and Social Networks.
    %__________________________________________________________________________________________________________________
    % Education
    \section{\mysidestyle Education}

    \textbf{University of Southern California}, Los Angeles, California, USA \\%\vspace{2mm}\\\vspace{1mm}%
    \textsl{Doctor of Philosophy (PhD), Electrical Engineering} \hfill \textbf{ Aug 2008 -- Present}\vspace{-3mm}\\\vspace{-1mm}
    \begin{list2}
        \item GPA: 3.93/4
        \item Advisors:  Prof. Bhaskar Krishnamachari \& Prof. Alexandros G. Dimakis
    \end{list2}\vspace{-1.5mm}
    
    
    \textbf{Indian Institute of Technology (IIT)}, Kharagpur, West Bengal, India \\%\vspace{2mm}\\\vspace{1mm}%
    \textsl{B.Tech(H), Electronics and Electrical Communication Engineering} \hfill \textbf{ July 2004 -- May 2008}\vspace{-3mm}\\\vspace{-1mm}%
    \begin{list2}
    		\item GPA:	9.27 out of 10.00
    		\item Ranked 3/50 in the department and 7/650 in the Institute        
    \end{list2}\vspace{-1.5mm}%


\section{\mysidestyle Publications}
\begin{list2}  
   \item M. Sathiamoorthy, M. Asteris, D. Papailiopoulos, A. G. Dimakis, R. Vadali, S. Chen, D. Borthakur,
      ``XORing Elephants: Novel Erasure Codes for Big Data", \textsl{Accepted for publication, VLDB 2013}.
      \item M. Sathiamoorthy, K. R. Moghadam, B. Krishnamachari, F. Bai, ``Helper Node Allocation Strategies for Content Dissemination in Intermittently Connected Mobile Networks", in submission.
     \item M. Sathiamoorthy, A. G. Dimakis, B. Krishnamachari, F. Bai, 
     ``Distributed Storage Codes Reduce Latency in Vehicular Networks'', \textsl{Accepted for publication, Transactions on Mobile Computing 2013.}
      \item M. Sathiamoorthy, W. Gao, B. Krishnamachari, G. Cao,
    ``Minimum Latency Data Diffusion in Intermittently Connected Mobile Networks'', in
    \textsl{2012 IEEE 75th Vehicular Technology Conference: VTC2012-Spring, 6-9 May 2012, Yokohama, Japan.}
    \item M. Sathiamoorthy, A. G. Dimakis, B. Krishnamachari, F. Bai, 
    ``Distributed Storage Codes Reduce Latency in Vehicular Networks'', in
    \textsl{Proceedings of the IEEE INFOCOM Mini-conference, 2012.}
    \item M. Alresaini, M. Sathiamoorthy, B. Krishnamachari, M. J. Neely, ``Backpressure with Adaptive Redundancy (BWAR)'', 
    in \textsl{Proceedings of the IEEE INFOCOM, 2012.}
    \item S. Lee, S. Pattem, M. Sathiamoorthy, B. Krishnamachari, A. Ortega, ``Spatially-Localized Compressed Sensing and Routing in Multi-Hop Sensor Networks'', in \textsl{3rd International Conference on Geosensor Networks}, July 2009, Pages 11-20.  
    \end{list2}

\section{\mysidestyle Internship\\Experience}

    \textbf{Symantec Research Labs}, Culver City, CA\\
    \textsl{Summer Intern} \hfill May 2013 -- Aug 2013
	
    \textbf{General Motors R\&D}, Warren, MI\\
    \textsl{Summer Intern} \hfill May 2011 -- Aug 2011

    \textbf{University of Southern California}, Los Angeles\\%\vspace{1mm}\\\vspace{1mm}%
    \textsl{Summer Intern} \hfill May 2007 -- July 2007

    \textbf{Nanyang Technological University}, Singapore \\%\vspace{1mm}\\\vspace{1mm}%
    \textsl{Summer Intern} \hfill May 2006 -- July 2006

\section{\mysidestyle Research Projects}
  \textbf{Reliable Block Placement for Cold Data in Data Centers} 
  \begin{list2}
  \item In this ongoing work, I am investigating how best to store cold data in a data center with maximum efficiency while maintaining high reliability. To facilitate the study, I have designed a data center block storage simulator, and utilized Akka distributed computation framework to scale-out processing.
  \item Future work involves implementing a Block Placement Policy in Hadoop.
  \end{list2} 

\newpage
	
	\textbf{Novel Erasure Codes for Big Data}
  \begin{list2}
   \item In collaboration with Facebook, we implemented regenerating codes specially designed for data centers over Hadoop HDFS. About 2x reduction in network utilization and disk I/O during rebuilds. Paper accepted for publication in VLDB 2013.
   \item This Hadoop version is available at \href{https://github.com/madiator/HadoopUSC}{https://github.com/madiator/HadoopUSC}. 
  \end{list2}
	
	\textbf{Distributed Storage Coding for Vehicular Networks}
  \begin{list2}
   \item Erasure Coding applied to distributed storage in Vehicular Networks to minimize the delay in content retrieval. I obtained theoretical upper bounds on delay and showed regions where Network Coding performs better than naive distribution strategy. Simulated on real taxi trace datasets to show the improvement. Work published at Infocom-Mini 2012 and TMC 2013 (accepted).
   \item Continued the work at General Motors to test on \textsl{real vehicles}, where I developed a new inter-vehicle video sharing application based on GM's existing Wavecast system for vehicular communication. Additionally, developed an Android application to act as the front end (which connects to and controls the Linux based video sharing application wirelessly).
   \item I have set up \href{http://www.openvanet.org}{openvanet.org} where the code and the datasets are available to be downloaded.
   \end{list2}
      
  \textbf{Twitter Retweet Dynamics}
  \begin{list2}
   \item In a collaborative work with USC Annenberg School of Communication and Journalism, we collected twitter graph and retweet data and showed an interesting trend between the number of retweets received and the probability of retweeting by Twitter users. Work was presented at the \textsl{2nd Annual Annenberg Symposium} and was invited to present again at the \textsl{3rd Annual Annenberg Reception}.
  \end{list2}
  
  \textbf{Implementation of MapReduce}
  \begin{list2}
  \item Implemented MapReduce (a framework for distributed processing) on USC's High-Performance Computing and Communications (HPCC) cluster consisting of thousands of nodes, as part of a course. Used it to study large network datasets. 
  \end{list2}
    
  \textbf{Other projects} 
  \begin{list2}   
   \item Implementation of Nachos, a software that simulates a small OS, involving process management, memory management, interprocess communication, fault tolerance etc. for a course on Operating Systems.
   \item Flash Scheduling - proposed and analyzed a new scheduling algorithm in a multi-user communication system with varying number of users (term project for Computer Communications course).
   \item At University of Southern California (2007 Summer Internship), I worked on the energy reduction of Wireless Sensor Networks using Compressed Sensing. Compressed Sensing is used to integrate compression and sensing to achieve energy gains as high as 90\% in ideal conditions.
   \item At Nanyang Technological University (2006 Summer Internship), I developed an English Continuous Speech Recognizer based on TIMIT Database using HTK software. Worked on Variable Frame Rate Algorithms and tested them on the CENSREC-3 database. Researched on Spectral Entropy based Speech Features and came up with modifications along with testing it on the CENSREC-3 database.
  \end{list2}


 \section{\mysidestyle Talks \\ \& Posters} 
  \begin{list2}
  \item Coded Distributed Storage for Cloud Environments. Pitch at \textsl{MHI Research Festival 2013}. \textbf{Best Pitch Award} (Honorable Mention).  
   \item Helper Node Allocation Strategies for Content Dissemination in Intermittently Connected Mobile Networks. E-Poster at \textsl{5th Annenberg Symposium 2013}. 
   \item Distributed Storage Codes Reduce Latency in Vehicular Networks. Presentation at \textsl{Infocom 2012, Orlando, FL}. Poster at \textsl{MHI Research Festival 2011}.
   \item Minimum Latency Data Diffusion in Intermittently Connected Mobile Networks. Presentation at \textsl{VTC Spring 2012, Yokohoma, Japan}.   
   \item A Study of Twitter Retweet Dynamics. Presentation at \textsl{2nd Annenberg Symposium 2010}, invited again for \textsl{3rd Annenberg Fellows Reception}.
   \item XORing Elephants: Novel Erasure Codes for Big Data. Poster at MHI Research Festival 2012
   \item Reliable Storage for Digital Media. \textsl{E-Poster at 4th Annenberg Symposium 2012}. 
   \end{list2}	
   
    \newpage
      
 \section{\mysidestyle Skills} 

    \textsl{Programming Languages}: C, C++, Java\\
    \textsl{Software Experiences}: Hadoop, Android, TCP/IP, Matlab, \LaTeX, Processing\\
    %\textsl{Databases}: MySQL, MongoDB\\
    %\textsl{Scripting Languages}: Perl, Linux shell scripting \\ %too old
    %\textsl{Hardware Description}: Verilog HDL \\ %too old
    %\textsl{Web Technologies}: HTML, CSS, JavaScript, PHP \\ %too old
    %\textsl{Foreign Languages}: German [beginner level] %too old

%__________________________________________________________________________________________________________________

 % Courses

\section{\mysidestyle Teaching}
TA for \textbf{Wireless and Mobile Device Networks Design and Laboratory} (Spring 2012).

\section{\mysidestyle Courses} 

%\begin{tabular}{@{}p{10cm}p{4cm}}
%Analysis of Algorithms       \\  Approximation Algorithms(ongoing)                   \\
%Design and Analysis of Computer Communication Networks                        \\
%Computer Communications (ongoing) \\ Queueing Theory                         \\
%Probabilistic Methods in Computer Systems Modeling    \\
%Operating Systems       \\  Wavelets        \\
%\end{tabular}
Trends in Cloud Computing and Data Center Networking\\
\begin{tabular}{@{}p{7cm}p{7cm}}
Distributed Storage Theory & Algebraic Coding Theory \\
Analysis of Algorithms  &  Approximation Algorithms                   \\
Computer Communications  & Operating Systems \\
Random Processes in Engineering  & Wavelets \\
\end{tabular} \\
Queueing Theory \\
Design and Analysis of Computer Communication Networks \\
Probabilistic Methods in Computer Systems Modeling


\section{\mysidestyle Software \\ Releases}
\begin{list2}
\item VANETSim: A Vehicular Network Content Dissemination Simulator. Java based tool that can talk with real-vehicle GPS trace datasets and can simulate content dissemination between vehicles. Code and datasets available at \href{http://www.openvanet.org}{openvanet.org}.
\item HadoopUSC: The raid-contrib project implements new regenerating (erasure) codes. \\Available at \href{https://github.com/madiator/HadoopUSC}{github.com/madiator/HadoopUSC}.
\end{list2}


%__________________________________________________________________________________________________________________
    % Honours and Awards
    
\section{\mysidestyle Honors and\\Awards} 
\begin{list2}
	  \item USC Annenberg Fellowship 2008-2012	- University of Southern California
          \item InfoUSA Summer Fellowship 2007	- University of Southern California %\vspace{1mm}\\%   
	  \item One of the seven finalists of Trilogy's Pirates of the Corporate, a web 2.0 business plan contest held at Hong Kong. %\vspace{1mm}\\%       
	  \item Amateur Radio License (Grade II) from Ministry of Communications, Government of India (2003).%\vspace{1mm}\\%
	  \item Ranked among top 0.5\% of about 170,000 students appeared in JEE 2004. \\%\vspace{1mm}\\%
% \item Secured Rank 86 in TNPCEE (Tamilnadu Professional Courses Entrance Examination, the Common Entrance Test for Tamilnadu Engineering Colleges) out of more than 150,000 students.
\end{list2}
  

%______________________________________________________________________________________________________________________
\section{\mysidestyle References} 
Prof. Bhaskar Krishnamachari \\
Associate Professor \\
Ming Hsieh Department of Electrical Engineering \\
USC Viterbi School of Engineering, Los Angeles, CA \\
Email: bkrishna at usc.edu \\
\href{http://ceng.usc.edu/~bkrishna}{http://ceng.usc.edu/$\sim$bkrishna}

Prof. Alexandros G. Dimakis \\
Assistant Professor \\
Ming Hsieh Department of Electrical Engineering \\
USC Viterbi School of Engineering, Los Angeles, CA \\
Email: dimakis at usc.edu \\
\href{http://www-bcf.usc.edu/~dimakis}{http://www-bcf.usc.edu/$\sim$dimakis}

%______________________________________________________________________________________________________________________
\end{resume}
\end{document}



